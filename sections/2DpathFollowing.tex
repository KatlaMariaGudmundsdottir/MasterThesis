\setstretch{1.6}
\sectiontitle{6}{2D Path following}
\lhead{2D Path following} % section header

\subsection{Background}
\subsubsection{Path Following Definition}
Path following is a  motion control scenario where a “vessel” has to follow a predefined path without any time constraints \cite{caharija_integral_2016}. It thereby differs to trajectory tracking since the path is not parameterized by time, but rather by another useful parameter \cite{hung_review_2023}, such as path length. 

This distinction is important since in the application to brain surgery spatial control is a lot more critical than timing. Also since the system can make smoother corrections and prioritize accuracy over speed when there are no strict time requirements. Path following also allows smoother convergence to the path and can reduce actuator actuator activity, since it avoids the constant adjustments required by time-based tracking (Aguiar and Hespanha 2007).

\subsubsection{Application of marine path-following to ribbon navigation in brain tissue}
The steering of the tendon-driven ribbon in brain tissue can be modelled to e kinematacally analogous to that of a boat or UAV. In a marine vessel vessel, turning the helm deflects the rudder, inducin a curvature that realigns the hull with the desired path. In the surgical device, pulling on the tendons causes the flexible tip of the probe to bend in a specific direction. This bending changes the direction of the devices motion, in much the same way that a rudder steers a boat. 
\newline \newline 
Because the bending of the tip of the device directly controls the direction of motion and the relationship between curvature and lateral deviation is governed by the same kinematic principles as in boat steering, the curvature or heading angle computed by marine guiddance laws are directly applicable by converting the heading angle to target bending angle for the device. This target bending angle can then be used to compute the corresponding tendon tensions required to steer the probe along the desired path.
\newline \newline 
Therefore, guidance methods established in marine robotics are well-suited as a foundation for developing the path-following strategy in this application. The following sections therefore provide an overview of the guidance systems popularly employed and developed for marine navigation.

\subsubsection{Guidance systems}
Path following can further be divided into two parts: the guidance system and the control system \cite{qi_curve_2022}. Where the guidance system is responsible for determining the desired target which the control system can then uses in order to steer such that the "vessel" stays on the path, or is lead toward the path if the cross -track error (that is the the shortest distance to the path) is nonzero. Some of the most popular methods adopted by the marine community are line-of-sight(LOS) guidance, pure pursuit guidance, vector field guidance and constant bearing guidance \cite{qi_curve_2022}, \cite{lekkas_integral_2014}. 

\paragraph*{Pure pursuit (Orthogonal Projection)}
In the orthogonal projection/Pure pursuit method the point on path that is closest to the vehicle is used as the "reference point" and then steering towards the point. This leads the along-track error to always be zero and only the corss-track error need to be controlled. The method then computes the desired yaw rate in the case of boaths (i.e. how fast the vehicle should turn) to drive the cross-track error to zero. 
\newline \newline
This approach is simple and intuitive. However, the convergence behavior may be problematic in practic. Since the controller always aims for the nearest point on the path it continues to generate turning commands even when the vehicle is nearly aligned. This causes overshooting and unnecessary steering corrections. Further, this method can encounter issues in certain geometric scenarios. Particularly on circular paths when the vehicle crosses the centerpoint of the curvature. In this case the projection onto the path is no longer well-defined and the method can break down.

\todo{python figure of pure pursuit!}

\paragraph*{Line of Sight}
The LOS guidance method is now widely used due to its small computational cost, easy implementation and intuitiveness \cite{qi_curve_2022} \cite{caharija_integral_2016}. Instead of using the closest point on the path, LOS guidance steers the vehicle toward a point located a fixed distance ahead on the path (known as the lookahead distance). In effect it imitates a helmsman steering the vessel toward a point lying at a constant distance ahead of the ship along the desired path \cite{caharija_integral_2016}. In the case of no enivronmental disturbances, simple LOS has good path convergence properties \cite{borhaug_integral_2008}. 
\newline \newline
However, path following control approaches based on traditional LOS guidance are susceptible to environmental disturbances \cite{borhaug_integral_2008} or more generally all errors that case the actual navigation direction to differ from the guidance direction. In particular this will lead to path deviation and converegence problems \cite{borhaug_integral_2008}. Moreover, this drawback cannot be fixed by simply adding integral action to the heading controller, the problem originates from the heading reference generator, that is, the LOS guidance law itself \cite{borhaug_integral_2008}.

\paragraph*{Integral Line of Sight (ILOS)}
To adress the limitation of traditional LOS, Børghaug et al. \cite{borhaug_integral_2008} proposed a guidance law based on the LOS guidance principle but that includes integral action to overcome the drawbacks of environmental disturbance while preserving the intuition and the simplicity of traditional LOS guidance.
\newline \newline 
By integrating the cross-track error over time, ILOS effectively reduces steady-state error and restores path convergence even when the vessels actual movement direction deviates from its commanded heading.

\paragraph*{Adaptive Methods}
Adaptive guidance methods can improve path-following in uncertain or dynamic environments by adjusting parameters in real time. Several adaptive approaches have been adopted for path-following including adaptive LOS, where the lookahead distance is baried based on cross.track oerror or path curvature \cite{hung_review_2023}. This allows the system to steer more smoothly when far from the path and more precisely when close, reducing overshooting and unnecessary correction.
\newline \newline 
Other methods include adaptive vector field guidance which modifies the synthetic vector field used to steer the vehicle toward the path. By adjusting the shape or strength of the filed based on real-time errors or estimated disturbances \cite{soetanto_adaptive_2003}. Or other techniques including sliding mode control \cite{soetanto_adaptive_2003}, Learning-based MPC and reingorcement learning based control \cite{hung_review_2023}.

\paragraph*{Model Predicitve Control (MPC)}
Model Predictive Control (MPC) is an optimization-based approach that computes control actuation by predicting future system behavior over a finite time horizon. At each time step, it solves and optimal control problem that minimizes path-following error while satisfying system constraints such a speed, turnign limits, and actuator bounds \cite{hung_review_2023}. 
\newline \newline
Unlike geometric methods like LOS or pure pursuit, MPC can explicitly handle physical constraints and nonlinear dynamics. 

\subsubsection{Kinematic modeling}



\subsection{Methods}

\subsubsection{Kinematic modeling}

\subsubsection{Choice of Path-Following Method}
Since the main focus is to establish a simple and effective path following system the choice of path following strategy must be reliable, scalable and computationally simple solution. The initial navigation method is based on a simple LOS guidance law. LOS offers good convergence properties with minimal computational overhead, making it a good practical starting point.
\newline \newline 
While standard LOS doe snot handle disturbances the design is modular and can easily be extended to integral LOS (
ILOS) by adding an integral term to the heading reference. This will allow for compensation of path steady state errors and ensure quicker convergence.
\newline \newline
During the implementation of this path-following strategy it is crucial to ensure that it is implemented in a modular way such that it may be switched out by more complex methods as discussed in the background such as adaptive or MPC based methods if further robustness or error handling is necessary.

\subsubsection{Path-Following Implementation Methodology}
The path-following system is implemented using a modular architecture that separates the navigation guidance, pitch control and tendon tension concerns. The overall system architecture consists of five primary modules that work together to achieve path following 

\paragraph*{Guidance}
The \texttt{Guidance} class implements the core navigation algorithm that generates heading references based on the current tip position and the desired path.

\paragraph*{title}

\subsubsection{Pitch autopilot}
In order for the ribbon to follow the path, the tip must continuously align wiht the desired pitch angle \(\theta_{ref}\) computed by the lOS algorithm. However,the physical system is controlled through tendon tensions which induce bending at the distal tip. Therefore, an intermedediate control layer, a pitch autopilot, must be implemented to convert heading references into actuator-level commands.
\newline \newline
It was decided 

\subsubsection{Tuning}
the Ziegler-Nichols relay method assumes that:

The system has an oscillatory behavior when switching between min and max control effort.

The output returns toward the setpoint (or crosses it) once actuation is disabled, creating that nice sine-like signal from which to estimate the ultimate gain and period.

But in your case — tendon-actuated heading control — that doesn’t apply:

Once you've pulled the tendons to deform the tip and achieve a certain heading, the heading stays there unless you actively pull the other way.

There’s no natural restoring force — no spring-back, no drift back — so the "relay" logic of toggling output and watching the system cross the setpoint repeatedly breaks down.

This is more like trying to autotune a position-controlled robot arm joint where gravity and friction dominate and the joint stays where it's told.

\subsubsection{Validation of Path Following}


\subsection{Implementation}

\subsubsection{Ribbon kinematic notation and definitions}
As mentioned in the hardware section the ribbon device is moved up and down i.e. in the z-plane via control of the linear stage which is attatched at the proximal tip of the device. However movement in the x and y plane is controlled via controlling the tensions at the tendons which are attached to motors at the top and the tip of the ribbon at the bottom. When the ribbon is inserted in a brain phantom via the movement of the linear stage the direction it will move in will depend on the deflection of the tip. Which is in turn dependent on the tendon activation pattern which results in the beingind or twisting of the tip. Since the device is constrained by the brain phantom a change in tendon tensions will not result in large deflections in the entire shape such as in a traditional tendon driven continuum robot, rather only the tip will deflect, leaving us with a very different kinematics. This makes the kinematics more similar to steerable needles rather than continuum robots, and since a 3D kinematic model of a steerable needle can be assumed to be similar to that of an underactued underwater vehicle \cite{secoli_closed-loop_2013}, we can assume the same here.  \todo{explain somewhere what steerable needles are}


The kinematic modeling starts with the Euler angle representation, which describes the orientation between the global frane and the local body coordinate frame as three rotations: roll \(\\phi\), pitch \(\theta\) and yaw \(\psi\) which represent rotations about the local body cartesian axes \(x\), \(y\) and \(z\) respectively.

\todo{Figure: showing local body frame z, x, y and the three angles on the ribbon}

\subsubsection{Kinematic model}
In the 2D kinematic model we assume the global y-position to be constant and thus using no velocity in the y-position and the rotation revolves around the local y-axis. The rotational matrix thereby becomes
\begin{equation}
    \textbf{R} = \textbf{R}(\theta) = \begin{bmatrix}
        cos \theta   &   0   &   -sin \theta \\
        0            &   1   &   0\\
        sin \theta   &   0   &   cos \theta
    \end{bmatrix}
\end{equation}

Thereby the kinematic model becomes
\todo{decide if it should be called global or inertial frame, what is the difference?}
\todo{decide on v notation}
\begin{equation}
    \vec{v_g} = \textbf{R}(\theta) \vec{v_b}
\end{equation}
\begin{equation}
    \begin{bmatrix}
        \dot{x}(t)\\ \dot{y}(t)\\ \dot{z}(t)
    \end{bmatrix} = 
    \begin{bmatrix}
        cos \theta   &   0   &   -sin \theta \\
        0            &   1   &   0\\
        sin \theta   &   0   &   cos \theta
    \end{bmatrix}
    \begin{bmatrix}
        v_x^b \\ v_y^b \\ v_z^b
    \end{bmatrix}
\end{equation}
\begin{equation}
    \begin{bmatrix}
        \dot{x}(t)\\ \dot{y}(t)\\ \dot{z}(t)
    \end{bmatrix} = \begin{bmatrix}
        v_x^bcos \theta - v_y^b sin \theta\\
        v_y^b\\
        v_x^b sing \theta + v_z^b cos \theta
    \end{bmatrix}
\end{equation}
Where \(\vec{v_b}\) is the velocity of the tip of the ribbon given in the body frame. However since we are in the 2D case where \(v_y^b = 0\) we can simplify this to:
\begin{equation}
        \begin{bmatrix}
        \dot{x}(t)\\ \dot{z}(t)
    \end{bmatrix} = \begin{bmatrix}
        v_x^bcos \theta - v_y^b sin \theta\\
        v_x^b sing \theta + v_z^b cos \theta
    \end{bmatrix}
\end{equation}
We can further simplify this by assuming that 


\todo{Figure: same figure as in ribbon kinematic model page calle ribbon system 2D, theta}

In order to simplify the control problem of the tendon driven ribbon we look at what is possible to control. 



\todo{see steerable needles, see if they have better explanations}.

\subsubsection{Navigation}
In order to make the ribbon follow a path the tendon tensions must be continually adjusted in such a way that it results in 

In the most simple LOS implementation for the 2D ribbon path following 

Therea re two major stage of navigation: insertion of the device into the phantom and advancement inside the gel \cite{noseda_flat_2024}. 

\subsubsection{Pitch autopilot}
The pitch autopilot is a closed-loop control system that continuously regulates the pitch of the tip. The autopilot layer consists of three primary components: a pitch angle converter, a feedback controller, and a tension converter. Together, these components transform the LOS heading reference into actuator-level tendon tension setpoints.


\paragraph*{Pitch converter}
The first step in the autopilot is estimating the measured pitch angle, \(\theta_{meas}\), from vision data. This is simply done by computing the angle between the z-axis and the 3D direction vector defined by the detected tip and pre-tip positions. 
\begin{equation}
    \theta_{meas} = atan2(\vec{d}_x,\vec{d}_z)
\end{equation}
where \(\vec{d}_x\) and \(\vec{d}_z\) are the \(x\)- and \(z\)- components of the direction vector. This computation is implemented in the \texttt{PitchConverter} class,  which can be later changed with more sophisticated conversion strategies later on.

\paragraph*{Feedback controller}
The main component of the autopilot is the PI controller implemented in the \texttt{FeedbackController} class. The controller receives the pitch error, defined as 
\begin{equation}
    \theta_e = \theta_{ref} - \theta_{meas}
\end{equation}
and calculates control output \(u\) intended to minimize the error. Since the ribbon system is highly damped, derivative control is unnecessary and therefore a PI controller was implemented with anti-windup for the integral term.
\begin{equation}
    u(t) = K_p \theta_e(t) + K_i \int \theta_e(t)
\end{equation}
Since the pitch control looped is triggered asynchronously by new vision data and the arrival time fo these updates may vary the effective sampling time \(\Deltat\) is computed each time using system timestamps. This ensures the validityof the integral terms.
\newline \newline
The output of the controller is a scalar actuation variable that represents the differential tension required between the left and right tendon groups. This value is later converted into explicit tendon tension setpoints in the next stage of the autopilot


\paragraph*{Tension converter}
The final step in the control loop is converting the actuation variable that is defined as 
\begin{equation}
    u(t) = T_{3, 4} - T_{2,1}
\end{equation}
Where \(T_{3, 4}\) and \(T_{1, 2}\) signify the tension on tendons 3 and 4 on the right and 1 and 2 on the left respectively.

The conversion stage maps the scalar actuation variable to individual tendon tensions, while maintaining a baseline tension of 100mN on all tendons to prevent slack. The conversion logic is:
\begin{equation}
\begin{cases}
T_1 = T_2 = 100 \text{ mN}, \quad T_3 = T_4 = 100 + u & \text{if } u > 0 \
T_1 = T_2 = 100 + |u|, \quad T_3 = T_4 = 100 \text{ mN} & \text{if } u < 0 \
T_1 = T_2 = T_3 = T_4 = 100 \text{ mN} & \text{if } u = 0
\end{cases}
\end{equation}



\paragraph*{F}

\subsection{Results}


\subsection{Discussion}

\subsubsection{Pitch conversion improvements}

\subsub




