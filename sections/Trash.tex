
In order to make the ribbon follow a path the tendon tensions must be continually adjusted in such a way that it results in 

In the most simple LOS implementation for the 2D ribbon path following 

Therea re two major stage of navigation: insertion of the device into the phantom and advancement inside the gel \cite{noseda_flat_2024}. 

\subsubsection{Sensor}
The \texttt{sensor} class was restructured to encapsulate all low-level configuration and data handling required for each individual sensor. Upon initialization, it configures the ADC to sample at the correct timeslot, sets the analog input gain, and synchronizes data acquisition with a dedicated counter using external trigger modes. This ensures that each sensor samples consistently and at the intended \SI{1}{\kilo\hertz} rate. The class also handles voltage-to-force conversion using sensor-specific calibration constants, allowing for flexible reuse across different sensors with varying sensitivities. The readout function returns both the converted voltage and the corresponding timestamp, ensuring the measurements are correctly time-aligned with the control loop. By modularizing the sensor interface in this way, the implementation became more maintainable, reusable, and easier to debug during both development and calibration.

\subsubsection{PI controller}
The feedback controller runs on a timer loop synchronized with the sensor update rate. It is structured to remain modular and extendable, with clearly defined interfaces for input (force readings and setpoints) and output (motor commands), enabling easy replacement or augmentation with more advanced controllers (e.g., adaptive or model-based) in future work.


\paragraph*{Guidance}
The \texttt{Guidance} class implements the core navigation algorithm that generates heading references based on the current tip position and the desired path.

\paragraph*{title}

\subsubsection{Pitch autopilot}
In order for the ribbon to follow the path, the tip must continuously align wiht the desired pitch angle \(\theta_{ref}\) computed by the lOS algorithm. However,the physical system is controlled through tendon tensions which induce bending at the distal tip. Therefore, an intermedediate control layer, a pitch autopilot, must be implemented to convert heading references into actuator-level commands.
\newline \newline
It was decided 

\subsubsection{Tuning Methodology}
the Ziegler-Nichols relay method assumes that:

The system has an oscillatory behavior when switching between min and max control effort.

The output returns toward the setpoint (or crosses it) once actuation is disabled, creating that nice sine-like signal from which to estimate the ultimate gain and period.

But in your case — tendon-actuated heading control — that doesn’t apply:

Once you've pulled the tendons to deform the tip and achieve a certain heading, the heading stays there unless you actively pull the other way.

There’s no natural restoring force — no spring-back, no drift back — so the "relay" logic of toggling output and watching the system cross the setpoint repeatedly breaks down.

This is more like trying to autotune a position-controlled robot arm joint where gravity and friction dominate and the joint stays where it's told.



Globally, it is estimated that around 50 million people live with epilepsy \cite{noauthor_epilepsy_nodate} with up to 27\% being resistent to traditional drug treatment \cite{sultana_incidence_2021}. Additionally around 150 000 new cases of glioblastoma brain tumors, an aggressive and malignant form of cancer with a very low survival rate, are diagnosed annually \cite{walsh_chapter_2016} with around a quarter of those deemed too deep or risky to fully resect with traditional methods \cite{vivas-buitrago_influence_2021}. Movement disorders such as parkinsons that affect nearly 10 million people worldwide are now also using high precision deep brain stimulation as a standard treatment.

Automation technologies that reduce the reliance on specialist manual skills could help lower have the potential to contribute to lowering the number 5 million individuals suffering from treatable neurosurgical conditions that will never undergo therapueutic surgical intervention. Developing closed-loop control for these devices is therefore not only a technical goal but a step toward improving global access to safe and effective neurosurgical care. In regions with limited access to skilled surgeons, automation and reliable navigation technologies could expand the availability of advanced neurosurgical procedures, improving access to safe and effective care.

\textbf{}



\subsubsection{Closed-Loop Design}
This modular and layered control structure supports accurate and smooth path following, while being modular and scalable. 





\subsubsection{Pitch Tracking}


\subsubsection{LOS Path-Following}



\subsubsection{ILOS Path-Following}

This pre-computation approach ensures that the control system can anticipate required actuations rather than simply reacting to tracking errors. 

The pre-computed path should be possible to follow by the ribbon

This ribbon-specific adaptation ensures that the curvature commands produce the intended mechanical response regardless of individual device variations in stiffness, tendon routing, or assembly tolerances. The characterization-based approach allows the control system to compensate for these variations, improving the accuracy and consistency of curvature control across different ribbon assemblies and enabling precise path-following performance.

The 2D path comparison shows an almost perfect match between the actual and target paths, with negligible deviation. These results demonstrate that ILOS guidance provides precise and robust performance in straight-line tracking, benefiting from incremental corrections and a smoother control response.