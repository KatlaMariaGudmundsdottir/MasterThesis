
%=================================================================
%                           Start Document
%=================================================================


\setstretch{1.4}
\section*{Abstract}
Navigating biomedical instruments inside the brain remains challenging and high risk. Many neurological conditions, such as epilepsy, Parkinson's disease and glioblastoma, require highly precise interventions. However, existing rigid tools are restricted in their ability to navigate the delicate and structurally complex brain tissue, leaving some cases effectively inoperable. To address this the MICROBS lab has developed a novel, flexible ribbon-shaped device capable of agile 3D navigation through tissue, accessing previously unreachable regions and enabling multiple targets per insertion. Its flat geometry is also well-suited for integrating microscale sensors, drug delivery and electrical recording and stimulation, opening possibilites for future multifunctional devices. 

This thesis presents the design, implementation, and validation of a closed-loop navigation system for the ribbon device, integrating real-time camera-based tip tracking, path planning, closed-loop tendon tension control, pitch control and a novel application of Integral Line-Of-Sight (ILOS) guidance for accurate 2D path following. All implemented within a modular and scalable software framework.

Validation in a brain phantom demonstrated mean path-following errors as low as 0.08mm, surpassing the state-of-the-art in flexible needle steering (best reported mean error 0.3 mm) and far exceeding the 3-5mm errors of manual placement. These results confirm the validity of the kinematic modeling and curved ILOS guidance developed in this thesis and provide a clear proof of concept for precise, automated control of the ribbon device in soft tissue environments. 
%================================================================
%                           End Document
%================================================================
