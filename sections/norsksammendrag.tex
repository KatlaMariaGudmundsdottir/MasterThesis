\setstretch{1.4}
\section*{Sammendrag}

Navigering av medisinske instrumenter inne i hjernen er fortsatt svært utfordrende og høy risiko. Mange nevrologiske tilstander, som epilepsi, Parkinsons sykdom og glioblastom, krever svært presise inngrep. Likevel er dagens stive instrumenter begrenset i sin evne til å navigere i hjernens skjøre og komplekse struktur, noe som gjør enkelte inngrep praktisk talt umulige. For å møte disse utfordringene har MICROBS-laboratoriet utviklet en ny, fleksibel båndformet instrument som muliggjør smidig 3D-navigasjon gjennom vev, slik at tidligere utilgjengelige områder kan nås og flere mål kan behandles med én og samme innføring. Den flate geometrien egner seg dessuten svært godt for integrering av mikrosensorer, legemiddeladministrasjon, samt elektrisk opptak og stimulering, noe som åpner for fremtidige multifunksjonelle instrumenter.

Denne masteroppgaven presenterer utvikling, implementering og validering av et lukket kontrollsystem for navigering av bånd-instrumentet. Systemet integrerer sanntids kamerabasert positions-sporing, baneplanlegging, lukket sløyfekontroll av senespenning, og pitch vinkel, samt en ny anvendelse av Integral Line-Of-Sight (ILOS)-styring for nøyaktig 2D-banenavigasjon. Alt implementert i et modulært og skalerbart programvaresystem.

Validering i en hjernefantom-modell viste gjennomsnittlig baneavviksfeil ned til 0,08 mm, som er bedre presisjon enn den beste rapporterte nøyaktigheten for fleksibel nålestyring (0,3 mm) og er langt bedre enn manuell plassering, som typisk har feil på 3–5 mm. Resultatene bekrefter gyldigheten av den kinematiske modelleringen og den kurvede ILOS-navigasjonen utviklet i denne oppgaven, og gir et tydelig proof-of-concept for presis, automatisert styring av bånd-instrumentet i mykt vev.
